\section{Product Design Specification}

The \acf{PDS} is a description of what you intend to design and lists all the requirements that a design needs to achieve in order to meet the customers needs. 
There are many ways a \ac{PDS} can be collated and formed, and in this exercise we will be using the format as shown in \cref{tbl-pds} and contains the following information:


\begin{table}
  \small
\begin{tabular}{r p{0.6\textwidth}}
  No. & Requirement number or revision number. \\
  Date & Date captured. \\
  Source & Reference to the material used to generate the requirement. \\
  Requirement & A description of the requirement. \\
  Target & A quantitative or qualitative metric that should be attained by the design. \\
  Must/Wish & Whether the requirement must be achieved by the design or you wish the design to achieve it. \\ 
  Method of Evaluation & A description of how you have evaluated your design against this requirement (e.g.\ force/stress calculations, user survey and/or expert vote). \\
  Cross-reference & Position in the report that this requirement is discussed.
\end{tabular}
\end{table}

\begin{table*}[h!]
  \centering
  \caption{Product Design Specification Template}
  \label{tbl-pds}
  \begin{tabular}{l l l l l l l l}
    \toprule
    No. & Date & Source & Requirement & Target & Must/Wish & Method of Evaluation & Cross-Ref \\
    \midrule
    1 & & & & \\
    Rev 1a & & & & \\
    2 & & & & \\
    \ldots & & & & \\
    \bottomrule
  \end{tabular}
  \vspace{1em}
\end{table*}

\marginnote[3em]{garvins' dimensions of quality} In addition to this, there are a variety of ways one can breakdown the \ac{PDS} to ensure that all aspects are covered. This could be by component/sub-system or looking at the design from a number of different perspective. One such breakdown is Garvins' Dimensions of Quality\cite{garvin1987}, which are defined as:

\begin{table}
  \small
\begin{tabular}{r p{0.6\textwidth}}
  Performance & The operating characteristics of the design \\
  Features & Additional capabilities and characteristics of the design \\
  Reliability & Factors that effect the likelihood of failure of the design in a specific time period \\
  Conformance & The standards and rules to which a design must adhere (i.e.\ health and safety) \\
  Durability & Factors that affect the lifespan of the design \\
  Serviceability & Factors that affect the maintenance of the design \\
  Aesthetics & Factors relating to how the design looks \\
  Manufacture \& Assembly & Factors relating to how the design is made and assembled \\
\end{tabular}
\end{table}